% Created 2014-07-29 Tue 23:48
\documentclass[11pt]{article}
\usepackage[utf8]{inputenc}
\usepackage[T1]{fontenc}
\usepackage{fixltx2e}
\usepackage{graphicx}
\usepackage{longtable}
\usepackage{float}
\usepackage{wrapfig}
\usepackage{rotating}
\usepackage[normalem]{ulem}
\usepackage{amsmath}
\usepackage{textcomp}
\usepackage{marvosym}
\usepackage{wasysym}
\usepackage{amssymb}
\usepackage{hyperref}
\tolerance=1000
\usepackage{xeCJK}
\setCJKmainfont{SimSun}
\author{xingwenpeng}
\date{\today}
\title{c}
\hypersetup{
  pdfkeywords={},
  pdfsubject={},
  pdfcreator={Emacs 24.3.1 (Org mode 8.2.7b)}}
\begin{document}

\maketitle
\tableofcontents

\section{数据结构面试题}
\label{sec-1}
\subsection{第一篇}
\label{sec-1-1}
\subsubsection{把二元查找树转变成排序的双向链表}
\label{sec-1-1-1}
题目:输入一棵二元查找树,将该二元查找树转换成一个排序的双向链表。要求不能创建任何新的结点,只调整指针的方向

\begin{verbatim}
#include <stdio.h>
#include <stdlib.h>

struct BSTTreeNode
{
    int value;
    struct BSTTreeNode *l;
    struct BSTTreeNode *r;
};
struct BSTTreeNode *head = NULL;

struct BSTTreeNode*NODE(int item, struct BSTTreeNode *l, struct BSTTreeNode *r)
{
    struct BSTTreeNode *t = malloc(sizeof *t);
    t->value = item;
    t->l = l; t->r = r;
    return t;
}
struct BSTTreeNode *insert_node(struct BSTTreeNode *t, int value)
{
    if (t==NULL) return NODE(value, NULL, NULL);
    if      (value < t->value)
    t->l = insert_node(t->l, value);
    else if (value > t->value)
    t->r = insert_node(t->r, value);
    return t;
}
void treetodoublelist(struct BSTTreeNode *t)
{
    static struct BSTTreeNode *pcurrent = NULL;
    t->l = pcurrent;
    if (NULL == pcurrent)
    head = t;
    else
    pcurrent->r = t;
    pcurrent = t;
}
void in_ord(struct BSTTreeNode *t)
{
    if (t==NULL) return;
    in_ord(t->l);
    treetodoublelist(t);
    in_ord(t->r);
}
void in_ord_show(struct BSTTreeNode *t)
{
    if (t==NULL) return;
    in_ord_show(t->l);
    printf("%d\n", t->value);
    in_ord_show(t->r);
}
void show_list(struct BSTTreeNode *h)
{
    struct BSTTreeNode *p;
    if (h == NULL) return;
    for (p = h; p != NULL; p = p->r)
    printf("%d\n", p->value);

}
int main(int argc, char *argv[])
{
    struct BSTTreeNode *root = NULL;
    root = insert_node(root, 10);
    root = insert_node(root, 6);
    root = insert_node(root, 14);
    root = insert_node(root, 4);
    root = insert_node(root, 8);
    root = insert_node(root, 12);
    root = insert_node(root, 16);
    //中序打印排序树
    in_ord_show(root);
    //排序树转换为双向链表
    in_ord(root);
    //遍历双向链表
    show_list(head);
    return 0;
}
\end{verbatim}
\subsubsection{下排每个数都是先前上排那十个数在下排出现的次数}
\label{sec-1-1-2}
给你10分钟时间,根据上排给出十个数,在其下排填出对应的十个数   
要求下排每个数都是先前上排那十个数在下排出现的次数。   

上排的十个数如下:   
【0,1,2,3,4,5,6,7,8,9】

举一个例子,   

数值: 0,1,2,3,4,5,6,7,8,9   

分配: 6,2,1,0,0,0,1,0,0,0   

0在下排出现了6次,1在下排出现了2次,   

2在下排出现了1次,3在下排出现了0次\ldots{}.   

以此类推..  
% Emacs 24.3.1 (Org mode 8.2.7b)
\end{document}